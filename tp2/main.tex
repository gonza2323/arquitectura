\documentclass{article}
\usepackage[spanish]{babel}
\usepackage[a4paper,
            top=2cm,
            bottom=2cm,
            left=3cm,
            right=3cm,
            headheight=36pt,
            nomarginpar,
            includehead,
            includefoot,
            ]{geometry}
\usepackage{graphicx}
\usepackage{listings}
\usepackage{fancyhdr}
\usepackage{parskip}
\usepackage{courier}
\usepackage{xcolor}
\usepackage{minted}
\usepackage{enumitem}
\usepackage{sectsty}


\sectionfont{\fontsize{12}{15}\selectfont}
\definecolor{bg}{gray}{0.1}

% Opciones para entornos de código fuente
\setminted[nasm]{
    style=github-dark,
    bgcolor=bg,
    fontfamily=tt,
    fontseries=b,
    % breaklines=true,
    % linenos=true,
    % firstnumber=0,
    fontsize=\small,
    % frame=single,
    % samepage=true,
    % xleftmargin=20pt,
    % xrightmargin=20pt,
}


% Encabezado y pie de página
\fancyhf{}
\lhead{\includegraphics[height=32pt]{img/logo-uncuyo-fing.pdf}}
\rhead{ Licenciatura en Ciencias de la Computación \\
        Arquitectura de las Computadoras \\
        TP Nº 2: Lenguaje Ensamblador}
\rfoot{\thepage}
\pagestyle{fancy}


\begin{document}
{\centering
    {\bfseries\Large Universidad Nacional de Cuyo \par}
    \vspace{-0.2cm}
    {\bfseries\Large Facultad de Ingeniería \par}
    \vspace{-0.2cm}
    {\bfseries\Large Licenciatura en Ciencias de la Computación \par}
    \pagestyle{plain}
    \vfill
    \noindent\hrulefill \\
    {\scshape\Huge Trabajo Práctico N\textsuperscript{\Large\underline o} 4\par} % Título
    \vspace{0.5cm}
    {\scshape\Large Arquitectura de Computadoras \par}
    {\scshape\large Entrada - Salida \par}
    \vspace{0.5cm}
    {\scshape\Large 2024 \par} % Semestre
    \noindent\hrulefill \\
    \vspace{2cm}
    {\Large
    Adriano Santino Fabris\\
    Gonzalo Padilla Lumelli\\
    Mariano Robledo
    \par}
    \vfill
    \setcounter{page}{1}
    \newpage
}
\section*{1. a) Programa que sume dos datos}
\inputminted{nasm}{./code/1a.txt}
\section*{1. b) Programa que realice la suma y la resta con dos datos almacenados en memoria}
\inputminted{nasm}{./code/1b.txt}
\section*{1. c) Escribir un programa que compare dos números. Si son iguales el programa debe finalizar y si son distintos los debe sumar}
\inputminted{nasm}{./code/1c.txt}
\section*{1. d) Un programa que lea un dato e indique si es par o impar}
\inputminted{nasm}{./code/1d.txt}
\section*{1. e) Programa que indique el funcionamiento del stack (pila).}
\inputminted{nasm}{./code/1e.txt}
\end{document}
