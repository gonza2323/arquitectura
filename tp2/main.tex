\documentclass{article}
\usepackage[spanish]{babel}
\usepackage[a4paper,
            top=2cm,
            bottom=2cm,
            left=3cm,
            right=3cm,
            headheight=36pt,
            nomarginpar,
            includehead,
            includefoot,
            ]{geometry}
\usepackage{amsmath}
\usepackage{graphicx}
\usepackage{listings}
\usepackage{fancyhdr}
\usepackage{parskip}
\usepackage{courier}
\usepackage{xcolor}
\usepackage{minted}
\usepackage{enumitem}
\usepackage{sectsty}


\sectionfont{\fontsize{10}{15}\selectfont}
\definecolor{bg}{RGB}{35, 39, 46}

% Opciones para entornos de código fuente
\setminted[nasm]{
    style=github-dark,
    bgcolor=bg,
    fontseries=b,
    fontsize=\small,
    % linenos=true,
    % firstnumber=0,
    % frame=single,
    % samepage=true,
    % xleftmargin=20pt,
    % xrightmargin=20pt,
}

\setminted[text]{
    style=default,
    fontseries=n,
    fontsize=\small,
}


% Encabezado y pie de página
\fancyhf{}
\lhead{\includegraphics[height=32pt]{img/logo-uncuyo-fing.pdf}}
\rhead{ Licenciatura en Ciencias de la Computación \\
        Arquitectura de las Computadoras \\
        TP Nº 2: Lenguaje Ensamblador}
\rfoot{\thepage}
\pagestyle{fancy}


\begin{document}
{\centering
    {\bfseries\Large Universidad Nacional de Cuyo \par}
    \vspace{-0.2cm}
    {\bfseries\Large Facultad de Ingeniería \par}
    \vspace{-0.2cm}
    {\bfseries\Large Licenciatura en Ciencias de la Computación \par}
    \pagestyle{plain}
    \vfill
    \noindent\hrulefill \\
    {\scshape\Huge Trabajo Práctico N\textsuperscript{\Large\underline o} 4\par} % Título
    \vspace{0.5cm}
    {\scshape\Large Arquitectura de Computadoras \par}
    {\scshape\large Entrada - Salida \par}
    \vspace{0.5cm}
    {\scshape\Large 2024 \par} % Semestre
    \noindent\hrulefill \\
    \vspace{2cm}
    {\Large
    Adriano Santino Fabris\\
    Gonzalo Padilla Lumelli\\
    Mariano Robledo
    \par}
    \vfill
    \setcounter{page}{1}
    \newpage
}

\section*{1. a) Programa que sume dos datos}

\inputminted{nasm}{./code/1a.txt}

El resultado queda en el registro A.

\section*{1. b) Programa que realice la suma y la resta con dos datos almacenados en memoria}

\inputminted{nasm}{./code/1b.txt}

El resultado de la suma queda en la dirección suma: 0x04, y el de la resta en resta: 0x05.

\section*{1. c) Escribir un programa que compare dos números. Si son iguales el programa debe finalizar y si son distintos los debe sumar}

\inputminted{nasm}{./code/1c.txt}

El resultado de la suma queda en la dirección suma: 0x04.

\section*{1. d) Un programa que lea un dato e indique si es par o impar}

\inputminted{nasm}{./code/1d.txt}

El resultado se imprime en el output.

\section*{1. e) Programa que indique el funcionamiento del stack (pila).}

Implemento un programa con funciones que pueden retornar valores y aceptar argumentos, mediante el uso del stack. El programa calcula el factorial de un número $n$, e imprime la fórmula $n! = n \times (n-1) \times \dots \times 2 \times 1 =$ \mintinline{text}|factorial(n)| en la salida. Se tienen la función \mintinline{text}{main()}, la función \mintinline{text}{factorial(n)}, que es recursiva, y las funciones \mintinline{text}{printNo(n)} y \mintinline{text}{printChar(c)}, que imprimen un número y un caracter, respectivamente. La función \mintinline{text}{main()} hace llamados a todas las demás; y \mintinline{text}{factorial(n)} llama a \mintinline{text}{printNo(n)}, \mintinline{text}{printChar(c)}, y a sí misma por ser recursiva. 

Antes de llamar a una función, se reserva en el stack el espacio suficiente para el valor de retorno, excepto si no retorna nada, y luego se agregan también sus argumentos, si los requiere. Cada función puede también crear variables locales en su porción del stack para no perder sus datos al llamar a otra función.

\inputminted{nasm}{./code/1e.txt}

\section*{  2. a) Cargar números en las direcciones 60, 61, 62 y 63.\\
            Restarle una constante (por ejemplo el, hexadecimal 7).\\
            Transferir el resultado a las direcciones 70, 71, 72 y 73.}

\inputminted{nasm}{./code/2a.txt}

\section*{  2. b) Cargar N números (por ejemplo 16) a partir de la dirección 60.\\
            Terminar el ingreso de números, si ingresa un dato igual a 0.}

\inputminted{nasm}{./code/2b.txt}

\section*{  2. c) Cargar N números a partir de la dirección 60.\\
            Restarle una constante (por ejemplo el, hexadecimal 5).\\
            Terminar el ingreso de números, si el resultado de la resta es cero.}

\inputminted{nasm}{./code/2c.txt}

\section*{  2. d) Cargar la línea de memoria RAM desde la memoria 40 a la 4F con 16 datos y transferirlos a partir de la dirección de memoria 60.}

\inputminted{nasm}{./code/2d.txt}

\section*{  2. e) Ejemplo de Hello World en español. Cambiar la salida por: Hola Mundo. ¿Qué tal? Explicar y/o comentar el programa en español.}

\inputminted{nasm}{./code/2e.txt}

\end{document}
